\documentclass[12pt]{jhwhw}
\author{Ian Malerich}
\title{Com S 331: Homework 8}
\usepackage{amssymb, amsfonts, amsthm, mathtools, graphicx, breqn}
\usepackage{subfig, float, scrextend, setspace, soul}
\usepackage{tikz, subcaption, caption, graphviz}
\usetikzlibrary{arrows.meta, automata, positioning}

\onehalfspacing
\begin{document}
\raggedright

%% Problem 1
\problem{}

	Define a NPDA for the language $L = \{a^nb^m: m,n\in \mathbb{N}, n\geq 3, m> n\}$.

\solution
	The NPDA here is pretty simple, $n \geq 3$ so we need at minimum $3$ a's in our string, so read
	those first, we can then add as many a's as we want, so we optionally push those onto the stack.
	Once we are done reading those we need more b's, so start reading b's until we we have an empty 
	stack. Once the stack is empty we know number of a's is equal to number of b's, so if we read one
	our condition is met and we can enter our final state. From there, there is no further limit on the number
	of b's (other than they must be finite of course), so we can continue to read b's with no consequences
	(and no stack modifications) once in the final state.

	$$
	\begin{tikzpicture}[baseline=(current axis.outer east)]
		\tikzset{vertex/.style = {shape=circle, draw, minimum size=3.0em}}
		\tikzset{edge/.style = {-{>[scale=1.75]},> = latex'}}
		\tikzset{loop/.style = {looseness=7}}

		\node[vertex, initial, initial where=left] (s0) at (0, 0) {$s_0$};
		\node[vertex, minimum size=2.6em] (_F) at (2, -4) {};

		\node[vertex] (a1) at (4, 0) {$a_1$};
		\node[vertex] (a2) at (8, 0) {$a_2$};
		\node[vertex] (a3) at (12, 0) {$a_3$};
		\node[vertex] (b) at (8, -4) {$b$};
		\node[vertex] (F) at (2, -4) {$F$};

		\draw[-{>[scale=1.75]}] (s0) to node[midway, above] {$a$ : push $a$} (a1);
		\draw[-{>[scale=1.75]}] (a1) to node[midway, above] {$a$ : push $a$} (a2);
		\draw[-{>[scale=1.75]}] (a2) to node[midway, above] {$a$ : push $a$} (a3);

		\draw[-{>[scale=1.75]}] (a3) to[loop right] node[midway, right] {$a$ : push $a$} (a3);

		\draw[-{>[scale=1.75]}] (a3) to[bend left] node[midway, below right] {$b$ : pop $a$} (b);
		\draw[-{>[scale=1.75]}] (b) to[loop below] node[midway, below] {$b$ : pop $a$} (b);

		\draw[-{>[scale=1.75]}] (b) to node[midway, above] {$b$ : empty stack} (F);
		\draw[-{>[scale=1.75]}] (F) to[loop below] node[midway, below] {$b$} (F);

	\end{tikzpicture}
	$$

%% Problem 2
\problem{}

	Consider the language
	$$
	L = \{w \in \{a,b\}^* : \text{the longest run of $a$'s in $w$ is longer than any run of $b$'s in $w$}\}
	$$
	Prove that $L$ is not context-free.

\solution

	\begin{proof} \textit{by contradiction} \\
		Assume $L$ is context-free. 
		Let $p\geq 1$ be the pumping length for $L$ by the Pumping Lemma for Context-Free Languages. \\
		Define $w = b^{p-1}a^pb^{p-1}$, note that the longest run of a's $=p$ and the longest run of b's $=p-1$,
		thus $w\in L$. We now apply the pumping lemma on this string.
		Let $w=uvwxy$ by this fact.
		Note that as $|vwx|\leq p$ by construction of $w$, if $b\in vwx$ then all b's in $vwx$ must come
		from one side of side of the string $w$. This is because $a^p$ is long enough that $vwx$ will not
		be able to reach the other set of b's without violating $|vwx|\leq p$.
		Therefore, when we pump, we are only modifying one run of b's (or none at all).

		\bigbreak
		case $|vx|_a \geq |vx|_b$: \\
		\begin{addmargin}[1em]{}
			Pump for $n=0$, thus we are removing at least 1 a from our string, we may or may not be removing
			any b's, but if we are removing b's, we are only removing them from one side of our string, the
			other side will remain untouched. By removing at least one a, our longest run of a's is now maximally
			$p-1$. As noted before, one run of b's must be remaining untouched, thus our longest run of b's is
			still also $p-1$. This violates the assumptions of our language, thus 
			$uv^0wx^0y \not\in L$ contradicting the pumping lemma.
		\end{addmargin}
		case $|vx|_a < |vx|_b$: \\
		\begin{addmargin}[1em]{}
			Pump for $n=2$, we have added at least one additional b over a's, if we added $k$ a's,
			we know have a longest run of $p+k$, note that all a's are next to each other in our string.
			We have also pumped one side of b's, we have added at least $k+1$ b's, so this particular run
			of b's is now minimally of size $p-1+k+1=p+k$, thus our longest run of b's is now $p+k$ equal to our
			longest run of a's. This violates the assumptions of our language thus 
			$uv^2wx^2y \not\in L$ contradicting the pumping lemma.
		\end{addmargin}

		Both exhaustive cases lead to contradictions, thus our initial assumption must be incorrect and
		we conclude that $L$ is not context-free.
		
	\end{proof}

%% Problem 3
\problem{}

	Prove or disprove that the following language is context-free:
	$$
	L = \{\alpha 2\beta : \alpha,\beta \in 1(0+1)^*, [\alpha]_2 < [\beta]_2\}
	$$
	where $[x]_2$ is the numerical value of the string $x$ interpreted as a positive number in base 2.

\solution

	\begin{proof} \textit{by contradiction} \\
		Assume $L$ is context-free. 
		Let $p\geq 1$ be the pumping length for $L$ by the Pumping Lemma for Context-Free Languages. \\
		Define $w=10^{p-1}1^{p-1}0\ 2\ 10^{p-1}1^p$. 
		Note that $[\alpha]_2+1 = [\beta]_2 \Rightarrow [\alpha]_2 < [\beta]_2$ 
		therefore $w\in L$. We can now apply the pumping lemma to this string.
		Let $w=uvwxy$ by this fact. 
		\bigbreak
		Note that the number of bits in $\alpha=\beta=2p$. \\
		Further note that we must pump both sides of the string, if we pump only $\alpha$, we can pump
		it larger than the $\beta$ (as it will have more bits), if we pump only $\beta$, we simply pump it for
		$n=0$ and it will be less than $\alpha$ (as it will have less bits). We can extend this assumption to
		state that both sides of our pumping ($x$ and $v$) must be equivalent in size (and we must pump something)
		we will denote this size $k$, that is $|x|=|v|=k\geq 1$. Again, if this is
		not the case, we simply pump such that $\alpha$ has more bits, or pump $n=0$ such that $\beta$ has less.
		Also observe that $2$ cannot be pumped as the string necessarily must only have one 2. 
		From these observations, it must be the case that
		$2\in w$ (else we would only be pumping $\alpha$ or only $\beta$) 
		and $v\subset \alpha \land x\subset \beta$.
		We also cannot pump the initial 1 of $\beta$, as that will leave a $\beta$ starting with a 0, which is
		not allowed. To pump with a leading 1, we need to follow $v$ with a 1, but the next $1$ in $\beta$ cannot
		be reached, as we only have $p/2$ bits to work with ($|xv|\leq p \land |x|=|v|\Rightarrow |v|\leq p/2$) 
		in $\beta$ which has $p-1$ 
		0's before 
		that $1$ can be reached.
		\bigbreak
		Note, that as $|vwx|\leq p$ we must then be pumping $1^{k-1}0$ in $v$ and $0^{k}$ in $x$.
		Therefore, after pumping n times, our string now looks like
		$$
			10^{p-1}1^{p-k}(1^{k-1}0)^n\ 2\ 1(0^k)^n0^{p-k-1}1^p
		$$
		Our strings still have the same length, but $\alpha$ has $p-1$ leading 0's (after the initial 1 bit), 
		where as $\beta$ now has $kn+p-k-1=k(n-1)+p-1$, thus for $n>1$ because $\alpha$ and $\beta$ have the
		same number of bits, and $\alpha$ has it's second leading 1 bit before $\beta$ will, 
		$[\alpha]_2 > [\beta]_2$. Therefore the pumping lemma fails and the language is not context-free.
	\end{proof}

\end{document}
