\documentclass[12pt]{jhwhw}
\author{Ian Malerich}
\title{Com S 331: Homework 11}
\usepackage{amssymb, amsfonts, amsthm, mathtools, graphicx, breqn}
\usepackage{subfig, float, scrextend, setspace, soul}
\usepackage{tikz, subcaption, caption, graphviz, color}
\usetikzlibrary{arrows.meta, automata, positioning}

\newcommand{\hlred}[1]{{\sethlcolor{pink}\hl{#1}}}
\newcommand{\hlgreen}[1]{{\sethlcolor{lime}\hl{#1}}}

\onehalfspacing
\begin{document}
\raggedright

%% Problem 1
\problem{}

	Are the following languages Turing decidable, Turing acceptable but not Turing-decidable,
	or not even Turing acceptable?

	\begin{itemize}
		\item $L=\{p(M)p(w): M 
			\text{ uses a finite number of tape cells when running on input }w\}$.
		\item $L=\{p(M)p(w)01^n0: M 
			\text{ uses at most $n$ tape cells when running on input}w\}$.
	\end{itemize}

	Here, ``using $n$ cells" means that the head of the (deterministic) TM $M$ reaches
	the $n$-th cell from the left during its computation. Justify your answer clearly:
	both exercises require careful thinking.

\solution

%% Problem 2
\problem{}

	Are the following languages Turing-decidable? Turing-acceptable but not Turing-decidable?
	Not even Turing-acceptable? For each answer, just give an intuitive explanation of
	your reasoning, no formal proof is required (just as in class, $M$ is a generic
	deterministic Turing machine, $w$ is a generic input string to it, and $p$ is an
	encoding function).

\solution

	In all the problems below, I discuss using a counter to keep track of iterations run by $M$, 
	this counter is stored at some arbitrary point on the tape. Not that important, but maybe worth mentioning.

	\part $$L = \{p(M) : |L(M)| \leq 10\}$$

	\textbf{acceptable} 
	Assume we are given a machine for a language such that $|L(M)| \leq 10$.
	We want to verify this condition given only the machine $M$. To do so would requiring enumerating
	all possible strings and checking whether or not $M$ accepts them, if so we would increment a counter.
	After enumerating all strings we would check that counter $\leq 10$ and if so we would accept.
	Note that this requires checking \textbf{all} possible strings and this is the problem. Strings
	can be arbitrarily large to infinite (though not infinite) and thus we cannot possible check all strings.
	Therefore \hlred{$L$ is not Turing-acceptable and cannot possibly then be Turing-decidable}.

	\part $$L = \{p(M) : |L(M)| \geq 10\}$$

	\textbf{acceptable}
	Assume we are given a machine for a language such that $|L(M)| \geq 10$.
	We have a finite alphabet we are working with that is encoded into $p(M)$.
	We can enumerate every possible string for sizes counting from $0,1,2,...$ and so on
	keeping a counter on when we find a string accepted by M. Once this counter reaches 11,
	we halt and accept. Because we know $M$ has at least $11$ such strings, and these strings
	have finite length, we must therefore be guaranteed to halt in finite time. 
	Therefore \hlgreen{$L$ is Turing-acceptable}.

	\bigbreak
	\textbf{decidable}
	We employ a similar approach here, this time given a machine such that $|L(M)| < 10$.
	Our machine must verify this condition given only $M$. However doing this would require
	permuting all possible input strings and making sure less than 10 are produced by $M$, if
	so our machine would halt and reject. However input strings can be arbitrarily large to infinite,
	thus our machine will loop forever checking every possible string.
	Therefore \hlred{$L$ is \textbf{not} Turing-decidable}.

	\part $$L = \{p(M)p(w) : M \searrow w \text{ in 10 steps or less }\}$$

	\textbf{acceptable}
	Assume we are given a machine $M$ and string $w$ and assume that the condition ``$M\searrow w \text{ in 10 
	steps or less}$" is true. This is easy to verify, simply run $M$ on $w$ keeping a counter on each step, 
	once our counter reaches 11 we halt and reject. If $M$ halts before that our machine will halt and accept.
	Because it is given that $M$ will halt in 10 steps or less on $w$, we know that our machine $L$
	will halt as well (and then accept).
	Therefore \hlgreen{$L$ is Turing-acceptable}.

	\bigbreak
	\textbf{decidable}
	Assume we are given a machine $M$ and string $w$ for which $M$ will not halt in 10 steps or less, this could
	mean that it halts in finite time or runs indefinitely. $L$ should reject this input. Again, following
	the outline given above, we will halt and reject as soon as our counter reaches 11, thus even if $M$
	does not halt on $w$ we don't care as $L$ won't need to run (simulate) past step 11 (of $M$).
	Therefore \hlgreen{$L$ is Turing-decidable}.

	\part $$L = \{p(M)p(w) : M \searrow w \text{ in 10 steps or more }\}$$

	\textbf{acceptable}
	Assume we are given a machine $M$ and a string $w$ for which $M$ will halt in 10 steps or more. This
	will be some finite positive integer, call it $n\geq 10$. Our machine $L$ will run $M$ on w and keep
	a counter on the number of iterations if $M$ halts and the counter is less than 10 we halt and reject
	otherwise we halt and accept. We know that $M$ will halt and specifically will halt with our counter 
	set to $n$. Further we know that $n\geq 10 \Rightarrow \text{counter}\geq 10$, thus our machine
	is guaranteed to halt and accept the input of $M$ and $w$.
	Therefore \hlgreen{$L$ is Turing-acceptable}.

	\bigbreak
	\textbf{decidable}
	If our machine $M$ halts on $w$ and does so in less than 10 steps we will obviously not have troubles.
	However the case of $M$ not halting on $w$ is trouble. In this case in fact we just have the halting problem.
	We cannot decide whether or not $M$ will halt in finite time using $L$. Therefore \hlred{$L$ is \textbf{not}
	Turing-decidable}.

%% Problem 3
\problem{}

	Use reduction to prove that the language 
	$$
		L = \{p(M_1)p(M_2): L(M_1) \subseteq L(M_2)\}
	$$
	is not decidable ($M_1$ and $M_2$ are Turing machines, of course).

\solution

\end{document}
