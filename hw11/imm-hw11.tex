\documentclass[12pt]{jhwhw}
\author{Ian Malerich}
\title{Com S 331: Homework 11}
\usepackage{amssymb, amsfonts, amsthm, mathtools, graphicx, breqn}
\usepackage{subfig, float, scrextend, setspace, soul}
\usepackage{tikz, subcaption, caption, graphviz, color}
\usetikzlibrary{arrows.meta, automata, positioning}

\newcommand{\hlred}[1]{{\sethlcolor{pink}\hl{#1}}}
\newcommand{\hlgreen}[1]{{\sethlcolor{lime}\hl{#1}}}

\onehalfspacing
\begin{document}
\raggedright

%% Problem 1
\problem{}

	Are the following languages Turing decidable, Turing acceptable but not Turing-decidable,
	or not even Turing acceptable?

\solution

	\part
		$$L=\{p(M)p(w): M 
			\text{ uses a finite number of tape cells when running on input }w\}.$$

	\textbf{acceptable} Assume we are given some machine $M$ and a string $w$ for which our condition holds, that
		is $M$ uses a finite number of tape cells when running on input $w$. We want $L$ to halt and accept
		this input configuration. Note that it is not necessary for $M$ to halt, it is possible that $M$ loops
		over a finite subset of the tape, thus never halting, but never using infinite tape cells.
		If this is the case (finite memory), note that Q, $\Sigma$ are both finite, the tape head can only
		be on a finite number of positions, and the tape stores a finite amount of data. Given this, we can
		produce some sort of finite 2-tuple (machine state and the finite state of the tape) 
		which encodes \hl{every possible state this machine can be in} (I reiterate
		we are using the strong assumption that $M$ uses finite resources). Since we know that there are finite
		such possible 2-tuples (there is a finite number of unique configurations our machine can be in), and
		that the machine never halts, we can conclude that the machine will enter the exact same configuration
		at least twice (infinitely in particular) for at least one configuration of the machine, this follows from
		the Pigeon-Hole Principle (infinitely iterations cannot fit in a finite number of states).
		Further if a machine enters a given state, call it $S$ and later enters that same exact state $S$, we can
		guarantee that the machine will infinitely enter state $S$ as it runs. This is because our 
		current configuration is exactly the same and one run following $S$ led to another $S$, 
		all subsequent occurrences of $S$ will lead to another. 
		\bigbreak
		From all this we conclude if the machine does not halt but uses
		finite resources, it MUST repeat some configuration of the machine (including state and tape layout).
		Since $S$ is repeated infinitely often, we know that it is impossible that we are writing to more
		cells in the tape then are used in $S$, else the next time we reach $S$ we would have a different tape
		configuration contradicting $S=S$ (that is repeated configurations $\Rightarrow$ finite memory).
		\bigbreak
		Given input $M$ and $w$ our machine will halt and accept if $M$ halts. If $M$ halts it must have done
		so in finite steps $\Rightarrow$ it used finite resources $\Rightarrow$ we should accept it. For each
		step of $M$ we will encode it's current configuration (as an 2-tuple) at some empty point on our tape.
		Then we will check all configurations this machine has been in (which will always be finite, but arbitrarily
		large) and if any configuration is repeated we will also halt and accept. This is because as argued
		above $M$ has entered an infinite loop via repeated configurations $\Rightarrow$ $M$ is using at most
		the number of cells used in that repeated configuration $\Rightarrow$ $M$ is using
		a finite number of cells. In both cases ($M$ halts and $M$ does not halt) $L$ will halt and accept.
		\bigbreak
		If $M$ uses an infinite number of tape cells, $L$ will not halt (thus will not accept), 
		this is argued below for decidability but mentioned to verify that $L$ does not accept these configurations.
		\hlgreen{Therefore $L$ is Turing-acceptable}.

	\bigbreak
	\textbf{decidable} Assume we are given some machine $M$ and a string $w$ for which $M$ uses an 
		infinite number of cells. As $M$ can only modify one tape cell in one step, use of infinite tape cells
		requires infinite steps $\Rightarrow$ $M$ does not halt. Further, we can make now assumptions about
		the behavior of $M$ with regards to not halting (above we could make assumptions about space usage). Thus
		we would need to be able to solve the halting problem in order to conclude to halt and reject $M$. But
		as the halting problem is undecidable, \hlred{$L$ must also be \textbf{not} decidable}.

	\part
		$$L=\{p(M)p(w)01^n0: M 
			\text{ uses at most $n$ tape cells when running on input}w\}.$$

	\textbf{acceptable}
		Here we have a similar setup to part (a) with the following modifications: after each iteration of $M$,
		check the simulated tape for $M$, if the number of used tape cells exceeds $n$ (encoded as a string of 1's),
		then halt and reject. If $M$ halts, then it never exceeded $n$ tape cells (else we would have already
		halted and rejected), thus we can halt and accept. If $M$ repeats a configuration (and enters a loop as noted
		above), then we know that the number of currently used tape cells is the maximum it will ever use (argued
		in part (a)), therefore we can halt and accept (again if it was using more than n we would have already
		rejected it).
		\bigbreak
		Assume we are given $M$, $w$ and $n$ encode into $L$ and that $M$ computes $w$ using at most $n$ tape
		cells. Then, if $M$ halts our machine must also halt, and since it never used more than $n$ tape cells,
		we never rejected it, therefore we have accepted it. If $M$ does not halt, $n$ is finite $\Rightarrow$
		$M$ uses a finite amount of memory $\Rightarrow$ $M$ will enter a repeat configuration. That configuration
		will not use more than $n$ tape cells, therefore we will have never rejected the input and will instead
		halt and accept. In both cases, $L$ halts and accepts, therefore \hlgreen{$L$ is Turing-acceptable}.

	\bigbreak
	\textbf{decidable} \\
		\textit{case $M$ halts:} \\
			Since $L$ simply simulates $M$ with a couple of extra steps in between each iteration,
			if $M$ halts then so must $L$ this is fairly trivial. Further if $M$ has used more
			than $n$ tape cells, we must have rejected it whenever it past that barrier.
			$L$ halts and rejects in this case.
		\bigbreak
		\textit{case $M$ does not halt, uses finite tape cells:} \\
			$M$ must have entered a repeat configuration and we must have halted at latest at that point.
			If $M$ uses more than $n$ tape cells then we rejected as soon as it did so. Thus $L$ both
			halts and rejects in this case.
		\bigbreak
		\textit{case $M$ does not halt, uses infinite tape cells:} \\
			$M$ uses infinite resources therefore it must use exactly $n+1$ resources at some point.
			Further it reaches $n+1$ resources in finite time, if it reached it in infinite time that
			would contradict the idea that $M$ uses infinite tape cells. $L$ is designed to halt
			and reject as soon as $n+1$ tape cells are used, and as these are used in finite time.
			$L$ halts and rejects in finite time in this case.
		\bigbreak
		In all cases, $L$ halts and rejects. Therefore \hlgreen{$L$ is Turing-decidable}.

%% Problem 2
\problem{}

	Are the following languages Turing-decidable? Turing-acceptable but not Turing-decidable?
	Not even Turing-acceptable? For each answer, just give an intuitive explanation of
	your reasoning, no formal proof is required (just as in class, $M$ is a generic
	deterministic Turing machine, $w$ is a generic input string to it, and $p$ is an
	encoding function).

\solution

	In all the problems below, I discuss using a counter to keep track of iterations run by $M$, 
	this counter is stored at some arbitrary point on the tape. Not that important, but maybe worth mentioning.

	\part $$L = \{p(M) : |L(M)| \leq 10\}$$

	\textbf{acceptable} 
		Assume we are given a machine for a language such that $|L(M)| \leq 10$.
		We want to verify this condition given only the machine $M$. To do so would requiring enumerating
		all possible strings and checking whether or not $M$ accepts them, if so we would increment a counter.
		After enumerating all strings we would check that counter $\leq 10$ and if so we would accept.
		Note that this requires checking \textbf{all} possible strings and this is the problem. Strings
		can be arbitrarily large to infinite (though not infinite) and thus we cannot possible check all strings.
		Therefore \hlred{$L$ is not Turing-acceptable and cannot possibly then be Turing-decidable} (note
		it cannot be decidable without first being acceptable).

	\part $$L = \{p(M) : |L(M)| \geq 10\}$$

	\textbf{acceptable}
		Assume we are given a machine for a language such that $|L(M)| \geq 10$.
		We have a finite alphabet we are working with that is encoded into $p(M)$.
		We can enumerate every possible string for sizes counting from $0,1,2,...$ and so on
		keeping a counter on when we find a string accepted by M. Once this counter reaches 11,
		we halt and accept. Because we know $M$ has at least $11$ such strings, and these strings
		have finite length, we must therefore be guaranteed to halt in finite time. Given a machine
		we would want to reject our machine will never halt (as argued below under decidability), thus this
		machine accepts the definition. Therefore \hlgreen{$L$ is Turing-acceptable}.

	\bigbreak
	\textbf{decidable}
		We employ a similar approach here, this time given a machine such that $|L(M)| < 10$.
		Our machine must verify this condition given only $M$. However doing this would require
		permuting all possible input strings and making sure less than 10 are produced by $M$, if
		so our machine would halt and reject. However input strings can be arbitrarily large to infinite,
		thus our machine will loop forever checking every possible string.
		Therefore \hlred{$L$ is \textbf{not} Turing-decidable}.

	\part $$L = \{p(M)p(w) : M \searrow w \text{ in 10 steps or less }\}$$

	\textbf{acceptable}
		Assume we are given a machine $M$ and string $w$ and assume that the condition ``$M\searrow w \text{ in 10 
		steps or less}$" is true. This is easy to verify, simply run $M$ on $w$ keeping a counter on each step, 
		once our counter reaches 11 we halt and reject. If $M$ halts before that our machine will halt and accept.
		Because it is given that $M$ will halt in 10 steps or less on $w$, we know that our machine $L$
		will halt as well (and then accept). If the machine halts in more than 10 steps, our machine
		will clearly reject it after it reaches 11.
		Therefore \hlgreen{$L$ is Turing-acceptable}.

	\bigbreak
	\textbf{decidable}
		Assume we are given a machine $M$ and string $w$ for which $M$ will not halt in 10 steps or less, this could
		mean that it halts in finite time or runs indefinitely. $L$ should reject this input. Again, following
		the outline given above, we will halt and reject as soon as our counter reaches 11, thus even if $M$
		does not halt on $w$ we don't care as $L$ won't need to run (simulate) past step 11 (of $M$).
		Therefore \hlgreen{$L$ is Turing-decidable}.

	\part $$L = \{p(M)p(w) : M \searrow w \text{ in 10 steps or more }\}$$

	\textbf{acceptable}
		Assume we are given a machine $M$ and a string $w$ for which $M$ will halt in 10 steps or more. This
		will be some finite positive integer, call it $n\geq 10$. Our machine $L$ will run $M$ on w and keep
		a counter on the number of iterations if $M$ halts and the counter is less than 10 we halt and reject
		otherwise we halt and accept. We know that $M$ will halt and specifically will halt with our counter 
		set to $n$. Further we know that $n\geq 10 \Rightarrow \text{counter}\geq 10$, thus our machine
		is guaranteed to halt and accept the input of $M$ and $w$. In the counter case, if $M$ halts
		in less than 10 steps, the counter will be less than 10 and we will thus appropriately reject. 
		If $M$ does not halt, as argued in decidability, $L$ will also not halt.
		Therefore \hlgreen{$L$ is Turing-acceptable}.

	\bigbreak
	\textbf{decidable}
		If our machine $M$ halts on $w$ and does so in less than 10 steps we will obviously not have troubles.
		However the case of $M$ not halting on $w$ is trouble. In this case in fact we just have the halting problem.
		We cannot decide whether or not $M$ will halt in finite time using $L$. Therefore \hlred{$L$ is \textbf{not}
		Turing-decidable}.

%% Problem 3
\problem{}

	Use reduction to prove that the language 
	$$
		L = \{p(M_1)p(M_2): L(M_1) \subseteq L(M_2)\}
	$$
	is not decidable ($M_1$ and $M_2$ are Turing machines, of course).

\solution

\end{document}
