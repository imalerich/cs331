\documentclass[12pt]{jhwhw}
\author{Ian Malerich}
\title{Com S 331: Homework 11}
\usepackage{amssymb, amsfonts, amsthm, mathtools, graphicx, breqn}
\usepackage{subfig, float, scrextend, setspace, soul}
\usepackage{tikz, subcaption, caption, graphviz}
\usetikzlibrary{arrows.meta, automata, positioning}

\onehalfspacing
\begin{document}
\raggedright

%% Problem 1
\problem{}

	Are the following languages Turing decidable, Turing acceptable but not Turing-decidable,
	or not even Turing acceptable?

	\begin{itemize}
		\item $L=\{p(M)p(w): M 
			\text{ uses a finite number of tape cells when running on input }w\}$.
		\item $L=\{p(M)p(w)01^n0: M 
			\text{ uses at most $n$ tape cells when running on input}w\}$.
	\end{itemize}

	Here, ``using $n$ cells" means that the head of the (deterministic) TM $M$ reaches
	the $n$-th cell from the left during its computation. Justify your answer clearly:
	both exercises require careful thinking.

\solution

%% Problem 2
\problem{}

	Are the following languages Turing-decidable? Turing-acceptable but not Turing-decidable?
	Not even Turing-acceptable? For each answer, just give an intuitive explanation of
	your reasoning, no formal proof is required (just as in class, $M$ is a generic
	deterministic Turing machine, $w$ is a generic input string to it, and $p$ is an
	encoding function).

	\begin{itemize}
		\item $\{p(M) : |L(M)| \leq 10\}$
		\item $\{p(M) : |L(M)| \geq 10\}$
		\item $\{p(M)p(w) : M \searrow w \text{ in 10 steps or less }\}$
		\item $\{p(M)p(w) : M \searrow w \text{ in 10 steps or more }\}$
	\end{itemize}

\solution

%% Problem 3
\problem{}

	Use reduction to prove that the language 
	$$
		L = \{p(M_1)p(M_2): L(M_1) \subseteq L(M_2)\}
	$$
	is not decidable ($M_1$ and $M_2$ are Turing machines, of course).

\solution

\end{document}
