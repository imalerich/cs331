\documentclass[12pt]{jhwhw}
\author{Ian Malerich}
\title{Com S 331: Homework 3}
\usepackage{amssymb, amsfonts, amsthm, mathtools, graphicx, breqn}
\usepackage{minted, subfig, float, scrextend, setspace, soul}
\usepackage{tikz, subcaption, caption}
\usemintedstyle{friendly}

\onehalfspacing
\begin{document}
\raggedright

%% Problem 1
\problem{}

	Prove or disprove: every finite language is recognized by some FA.

\solution
	
	\begin{proof} First note that L is finite, define $K = |L|$, the number
		of strings in L, K finite.
		As we have a finite number of strings, prescribe some arbitrary order to them,
		and assign each string some unique index $i$, where $1 \leq i \leq K$. Using this
		information, we can define an NFA which I claim will recognize the language.
		Start with some initial state $s_0$, if the empty string is in the language,
		$s_0$ is final, else it is not. Next, for each string, define a sequence
		of new states, starting with $s_0$ with each edge representing the next character
		in the string, the state will then represent the current progress into that string
		for that particular index. The terminating state of this sequence is final.
		By doing this, we form a tree starting at $s_0$ (which has K edges leaving it)
		with branches leaving $s_0$ each representing one of the (finite) 
		strings in the language. All leaves with no children are final states, $s_0$ may
		or may not be, and no other states are final. Finally, if there are any characters
		in the alphabet which do not have edges from $s_0$ add them all pointing to a new
		arbitrary non-final trap state. \\
		\bigbreak
		\textbf{Claim:} For every finite language, this algorithm produces an NFA
		representing it. \\
		Let $L$ be a finite language. \\
		Let $w \in L$, then w represents a branch of the NFA leading to a final state 
			$\Rightarrow$ $w$ is represented by the NFA. \\
		Let $x \not\in L$, if $x$ is empty, then it will never leave $s_0$, as $x\not\in L$
			then $s_0$ is not final, therefore $x$ is not represented by the NFA.
			If $x$ is not empty, it will travel down a path of our NFA, either the
			path leading to the trap state, or a path along a string. By construction
			of the NFA, $x$ must land on a non-final state, else it would be necessarily
			equivalent to a string in $L$ (contradiction).
		\bigbreak
		Therefore, as the NFA produced by the algorithm recognizes all strings in L, and 
		only strings in L, every finite language is recognized by the NFA generated
		by the algorithm $\Rightarrow$ every finite language is recognized by some FA.

	\end{proof}

%% Problem 2
\problem{}

	Define a DFA, simplified to the best of your abilities, that recognizes the
	language $L = \{w \in \{a,b\}^* : w \text{ does not contain the substring }
	abba \}$.

\solution
	
	The DFA for this is pretty straightforward, all states are final, until
	we find an occurrence of $abba$, in which case we hit a trap state.
	Each of the 4 other states represent how far away we are from discovering
	$abba$ (where $s0$ represents no progress).
	\bigbreak
	$$
	\begin{tikzpicture}[baseline=(current axis.outer east)]
		\tikzset{vertex/.style = {shape=circle, draw, minimum size=3.0em}}
		\tikzset{edge/.style = {->,> = latex'}}
		\tikzset{loop/.style = {looseness=7}}

		\node[vertex, minimum size=2.6em] (_s0) at (3, -3) {};
		\node[vertex, minimum size=2.6em] (_a) at (3, 0) {};
		\node[vertex, minimum size=2.6em] (_ab) at (6, 0) {};
		\node[vertex, minimum size=2.6em] (_abb) at (9, 0) {};

		\node[vertex] (s0) at (3, -3) {$s_0$};
		\node[vertex] (a) at (3, 0) {$a$};

		\node[vertex] (ab) at (6, 0) {$ab$};
		\node[vertex] (abb) at (9, 0) {$abb$};
		\node[vertex] (abba) at (9, -3) {$abba$};

		\draw[->] (abba) to[loop below] node[midway, below] {$a,b$} (abba);
		\draw[->] (a) to[loop above] node[midway, above] {$a$} (a);
		\draw[->] (s0) to[loop below] node[midway, below] {$b$} (s0);

		\draw[->] (s0) to node[midway, left] {$a$} (a);

		\draw[->] (a) to[bend left] node[midway, above] {$b$} (ab);
		\draw[->] (ab) to[bend left] node[midway, above] {$a$} (a);

		\draw[->] (ab) to node[midway, above] {$b$} (abb);
		\draw[->] (abb) to node[midway, right] {$a$} (abba);
		\draw[->] (abb) to[bend left] node[midway, above] {$b$} (s0);

	\end{tikzpicture}
	$$

%% Problem 3
\problem{}

	Conssider the $n$-bit binary representation of a natural number $x$:
	\begin{center}
		the binary representation of $x$ is $(x_{n-1}x_{n-2}\cdots x_1x_0)
		\Longleftrightarrow x = \sum_{i=0}^{n-1}x_i2^i$
	\end{center}
	where each bit $x_i$ is a binary digit, either zero or one. \\
	Consider the language
	\begin{align*}
		L = &\{a_0b_0c_0\cdots a_{n-1}b_{n-1}c_{n-1} : n \in \mathbb{N} \wedge \forall i,0\leq i
		< n, a_i \in \{0,1\}, b_i\in\{0,1\}, c_i\in\{0,1\} \wedge \\
		&(a_{n-1}\cdots a_0) + (b_{n-1}\cdots b_0)_2 = (c_{n-1}\cdots c_0)_2\}
	\end{align*}
	Define a DFA that accepts L.

\solution

	TODO

\end{document}
