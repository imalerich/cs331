\documentclass[12pt]{jhwhw}
\author{Ian Malerich}
\title{Math 331: Homework 1}
\usepackage[demo]{graphicx}
\usepackage{amsmath, amsfonts, amsthm, amssymb, scrextend, setspace, framed}
\usepackage{tikz, subcaption, caption}
\usetikzlibrary{arrows, positioning}
\onehalfspacing

\begin{document}
\raggedright{}

%% Problem #01
\problem{}

	Give a regular expression, simplified to the best of your abilities, for the language of \textbf{all} strings
	of \textit{a}'s, \textit{b}'s, and \textit{c}'s where \textit{a} is never immediately followed by \textit{b}.

\solution

	%% any a, has to have another a or c following it,
	%% else, it is the end of the string

	%% however, b and c may both be followed by any character

	$$
		(
		(a^+c^+)^*
		b^*c^*
		)^*a^*
	$$

	\textbf{Reasoning}

	Given any \textit{a}, it must be followed by another \textit{a}, a \textit{c} or be the end of the string.
	Thus, the first part of the expression $(a^+c^+)*$ covers the case of a string of \textit{a}'s which
	is not at the end of the string and thus must be followed by a c.
	\\
	The case of a string of \textit{a}'s terminating the string is covered at the end of the expression.
	\\
	Note that with the expression $(a^+c^+)^*a^*$ we have the set of strings, which if they are not empty,
	start in \textit{a} and only contain \textit{a} and \textit{c}.
	\\
	We simply follow our optional \textit{a}'s (followed by \texti{c}'s) with as many \textt{b}'s and 
	\textit{c}'s as we want. Note that they may be in any order, as all of our our groups are zero or more
	inside another group of zero or more.
	\\
	Further note that it is perfectly possible for a \textit{b} to be followed by an \textit{a}, this is fine
	and does not violate the terms of the problem.
	\\
	Thus, although \textit{c}'s are required following occurences of \textit{a} in the string, we can add
	individual \textit{c}'s in at any time with $\textit{c}^*$ (group 3 in the examples below).

	\bigbreak
	\textbf{Examples} (note groupings are not necessarily unique) \\
	Subscripts denote which group is used to generate each substring $((a^+c^+)_1^*b^*_2c^*_3)^*a^*_4$. \\
	\begin{align*}
		a &: a_4 \\
		b &: b_2 \\
		ac &: (ac)_1 \\
		acacbaaa &: (ac)_1(ac)_1b_2(aaa)_4 \\
		cccccbbb &: (ccccc)_3(bbb)_2 \\
		acbacb &: (ac)_1b_2(ac)_1b_2 \\
		cbacba &: c_3b_2(ac)_1b_2a_4 \\
	\end{align*}

%% Problem #02
\problem{}

	Give a regular expression, simplified to the best of your abilities, for the language of \textbf{all} strings
	of \textit{a}'s, \textit{b}'s, and \textit{c}'s that contain an even number of \textit{b}'s.

\solution

	$$
		(a+c+(b(a+c)^*b))^*
	$$

	\textbf{Reasoning}

	If we read strings from this language left to right, any \textit{b} which we find must be followed by 
	another \textit{b} separated only be optional \textit{a}'s and \textit{c}'s. 
	That is to say that, we can assume there will be 
	no \textit{b}'s between them, else we would have chosen the closer pair.
	\\
	Thus the problem can be thought of as the strings $(a+c)^*$, 
	with optional substrings surrounded by \textit{b}'s.

	\bigbreak
	\textbf{Examples} (note groupings are not necessarily unique) \\
	Subscripts denote which group is used to generate each substring $(a_1+c_1+(b(a+c)_3^*b)_2)^*$ \\
	\begin{align*}
		aaaccc &: 
			(aaa)_1(ccc)_1 \\
		baaccaab &: 
			(b(aaccaa)_3b)_2 \\
		bbbbbb &: 
			(bb)_2(bb)_2(bb)_2 \\
		aacbbbbabbacac &: 
			(aa)_1(c)_1(bb)_2(bb)_2(a)_1(bb)_2(a)_1(c)_1(a)_1(c)_1 \\
	\end{align*}

%% Problem #03
\problem{}

	Simplify (if possible) the expression $(a+b+c)^*(a+b)^*$, then describe as concisely as you can
	in English the language it defines.

\solution

	$$
		(a+b+c)^*
	$$

	Note that + takes the union of the left and right, in this case just the characters, 
	thus it behaves like an 'or' operator.
	By this logic, we know that $(a+b) \subset (a+b+c)$.
	Thus any string in $(a+b)*$ could instead be created by another application of $(a+b+c)*$.
	\bigbreak
	This expression defines a language of string containing the characters $\{a,b,c\}$ (in any order, combination
	and quantity).

%% Problem #04
\problem{}

	Simplify (if possible) the expression $(a+b)^*c^*(a+b)^*$, then describe as concisely as you can
	in English the language it defines.

\solution

	Cannot be simplified.
	\bigbreak
	This expression defines the language containing the characters $\{a,b,c\}$ where all characters \textit{c} must
	be neighbors of each other, with strings $\{a,b\}^*$ on either side.
	Note that the given expression describes exactly that, with no redundancy, 
	and no part can be removed without removing one of those properties.

%% Problem #05
\problem{}

	Define a DFA, simplified to the best of your abilities, for the language of \textbf{all} strings of 
	\textit{a}'s, \textif{b}'s, and \textif{c}'s where \textif{a} is never immediately followed by \textif{b}.

\solution

	$$
		M = (\{s_0,s_1,s_2\},\{a,b,c\},\delta, s_0, \{s_0,s_1\})
	$$

	Where $\delta: \{s_0, s_1, s_2\} \times \{a,b,c\} \rightarrow \{s_0,s_1,s_)\}$ is defined as follows: \\
	\begin{figure}[htp]
		\centering
		\begin{tabular}{||c|c c c||}
			\hline
			\delta & a & b & c \\ \hline
			s_0 & s_1 & s_0 & s_0 \\ \hline
			s_1 & s_1 & s_2 & s_0 \\ \hline
			s_2 & s_2 & s_2 & s_2 \\ \hline
		\end{tabular}
	\end{figure}

	As can be observed from the graph below, there are two final states which are considered valid,
	the first $s_0$ (also the initial state), is for any string which does not end in \textit{a}.
	The second, $s_1$, represents that the most recent character input was an \textit{a}, note that
	the only arrows pointing to $s_1$ are for an input of \textit{a}. A string ending in \textit{a} is still
	valid, so $s_1$ is also a final state. That only leaves $s_2$ which serves as a trap state marking
	the string as invalid. This state can only be reached from $s_1$ by an input of \textit{b}, thus 
	represents an input of an \textit{a} immediately followed by \textit{b}.

	\textcolor[RGB]{220,220,220}{\rule{\textwidth}{0.5pt}}

	\begin{figure}[htp]
		\centering
		\begin{tikzpicture}[baseline=(current axis.outer east)]
			\tikzset{vertex/.style = {shape=circle, draw, minimum size=3.0em}}
			\tikzset{edge/.style = {->,> = latex'}}
			\tikzset{loop/.style = {looseness=5}}

			\node[vertex, minimum size=2.6em] (b0) at (0, 0) {$s_0$};
			\node[vertex, minimum size=2.6em] (b1) at (3, 0) {$s_1$};

			\node[vertex] (s0) at (0, 0) {$s_0$};
			\node[vertex] (s1) at (3, 0) {$s_1$};
			\node[vertex] (s2) at (6, 0) {$s_2$};

			\draw[edge] (s2) to[loop] node[midway, above] {$a,b,c$} (s2);
			\draw[edge] (s0) to[loop] node[midway, above] {$b,c$} (s0);
			\draw[edge] (s1) to[loop] node[midway, above] {$a$} (s1);

			\draw[edge] (s0) to[bend left] node[midway, below] {$a$} (s1);
			\draw[edge] (s1) to[bend left] node[midway, above] {$c$} (s0);
			\draw[edge] (s1) to[bend right] node[midway, above] {$b$} (s2);
		\end{tikzpicture}
		\caption{}
		\label{fig:p5state}
	\end{figure}

%% Problem #06
\problem{}

	Define a DFA, simplified to the best of your abilities, that recognizes the language
	$$
		L = \{w\in \{a,b\}^*: |w|_a \text{ mod } 3 = 0\}.
	$$

\solution

	$$
		M = (\{s_0,s_1,s_2\},\{a,b\},\delta, s_0, \{s_0\})
	$$

	Where $\delta: \{s_0, s_1, s_2\} \times \{a,b\} \rightarrow \{s_0,s_1,s_)\}$ is defined as follows: \\
	\begin{figure}[htp]
		\centering
		\begin{tabular}{||c|c c||}
			\hline
			\delta & a & b \\ \hline
			s_0 & s_1 & s_0 \\ \hline
			s_1 & s_2 & s_1 \\ \hline
			s_2 & s_0 & s_2 \\ \hline
		\end{tabular}
	\end{figure}

	Note that we needn't concern ourselves with the number of \textit{b}'s we find in our strings.
	The only determining factor is the number of \textit{a}'s found. Thus we have three states, corresponding
	to the modulo remainders (respect to 3) for the number of \textit{a}'s. Each \textit{a} will progress
	to the next state (effectively counting modulo 3). Any \textit{b}'s found along the way form a self loop,
	and have no impact on the result. The only valid final state is $s_0$ which represents
	$|w|_a \text{ mod } 3 = 0$.

	\textcolor[RGB]{220,220,220}{\rule{\textwidth}{0.5pt}}

	\begin{figure}[htp]
		\centering
		\begin{tikzpicture}[baseline=(current axis.outer east)]
			\tikzset{vertex/.style = {shape=circle, draw, minimum size=3.0em}}
			\tikzset{edge/.style = {->,> = latex'}}
			\tikzset{loop/.style = {looseness=6}}

			\node[vertex, minimum size=2.6em] (b0) at (-1.73205, 1.0) {$s_0$};

			\node[vertex] (s0) at (-1.73205, 1.0) {$s_0$};
			\node[vertex] (s1) at (1.73205, 1.0) {$s_1$};
			\node[vertex] (s2) at (0, -2) {$s_2$};

			\draw[edge] (s2) to[loop below] node[midway, below] {$b$} (s2);
			\draw[edge] (s0) to[loop left] node[midway, left] {$b$} (s0);
			\draw[edge] (s1) to[loop right] node[midway, right] {$b$} (s1);

			\draw[edge] (s0) to[bend left=45] node[midway, below=1mm] {$a$} (s1);
			\draw[edge] (s1) to[bend left=45] node[midway, above left] {$a$} (s2);
			\draw[edge] (s2) to[bend left=45] node[midway, above right] {$a$} (s0);
		\end{tikzpicture}
		\caption{}
		\label{fig:p5state}
	\end{figure}

\end{document}
