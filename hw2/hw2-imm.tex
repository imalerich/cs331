\documentclass[12pt]{jhwhw}
\author{Ian Malerich}
\title{Math 331: Homework 1}
\usepackage[demo]{graphicx}
\usepackage{amsmath, amsfonts, amsthm, amssymb, scrextend, setspace, framed, tikz, caption, anyfontsize}
\usepackage[lofdepth,lotdepth]{subfig}
\usetikzlibrary{arrows, positioning}
\onehalfspacing

\begin{document}
\raggedright{}

%% Problem #01
\problem{}

	Define a DFA, simplified to the best of your abilities, to recognize the language
	$L = \{a^ib^j : (i+j)\text{ mod }3 = 0\}$.

\solution

	Note that the length of a string in the language will be (i+j), thus
	the specification of the language is simply looking for an sequence of a's
	followed by b's, such that the string has a length that is a multiple of 3.

	$$
		M = (\{a_0,a_1,a_2,b_0,b_1,b_2, T\},\{a,b\},\delta, a_0, \{a_0,b_0\})
	$$

	Where $\delta: \{a_0, a_1, a_2, b_0, b_1, b_2, T\} \times \{a,b\} \rightarrow \{a_0,a_1,a_2,
	b_0, b_1, b_2, T)\}$ 
	is defined by the following directed graph: \\
	$$
	\begin{tikzpicture}[baseline=(current axis.outer east)]
		\tikzset{vertex/.style = {shape=circle, draw, minimum size=2em}}
		\tikzset{edge/.style = {->,> = latex'}}
		\tikzset{loop/.style = {looseness=6}}

		\node[vertex, minimum size=1.6em] (BBB) at (2.25, -2.55) {};
		\node[vertex, minimum size=1.6em] (AAA) at (0, 3.897) {};

		\node[vertex] (a_1) at (-4.5, -3.897) {$a_1$};
		\node[vertex] (a_2) at (4.5, -3.897) {$a_2$};
		\node[vertex] (a_0) at (0, 3.897) {$a_0$};

		\node[vertex] (b_2) at (-2.25, -2.55) {$b_2$};
		\node[vertex] (b_0) at (2.25, -2.55) {$b_0$};
		\node[vertex] (b_1) at (0, 1.337775) {$b_1$};

		\node[vertex] (t) at (0, -1) {$T$};

		\draw[edge] (a_0) to node[midway, above left] {$a$} (a_1);
		\draw[edge] (a_1) to node[midway, above right] {$a$} (a_2);
		\draw[edge] (a_2) to node[midway, above right] {$a$} (a_0);

		\draw[edge] (a_0) to node[midway, right] {$b$} (b_1);
		\draw[edge] (a_1) to node[midway, above] {$b$} (b_2);
		\draw[edge] (a_2) to node[midway, above] {$b$} (b_0);

		\draw[edge] (b_0) to node[midway, above right] {$b$} (b_1);
		\draw[edge] (b_1) to node[midway, above left] {$b$} (b_2);
		\draw[edge] (b_2) to node[midway, below] {$b$} (b_0);

		\draw[edge] (b_0) to node[midway, right] {$a$} (t);
		\draw[edge] (b_1) to node[midway, right] {$a$} (t);
		\draw[edge] (b_2) to node[midway, left] {$a$} (t);

		\draw[edge] (t) to[loop below] node[midway] {$a,b$} (t);
	\end{tikzpicture}\hfill
	$$

%% Problem #02
\problem{}

	Describe in a short English sentence the language accepted by the DFA, and give
	a regular expression for it. Then, define a 5 state NFA that accepts the same language.

\solution

	The NFA defines strings of bits, who have a length of at least 4, and who's last 4 bits
	form a binary number between 0 and 7. That is to say, of the last 4 bits, the first
	will be 0, and the other 3 will be either 0 or 1.

	$$
	(0+1)^*0(0+1)(0+1)(0+1)
	$$

	We can define an NFA based on that regular expression\ldots \\

	$$
		M = (\{0+1_0,0,0+1_1,0+1_2,0+1_3\},\delta, \{0+1_0\}, \{0+1_3\})
	$$

	Where $\delta: \{0+1_0,0,0+1_1,0+1_2,0+1_3\} \times \{0,1\} \rightarrow 
	\{0+1_1,0,0+1_1,0+1_2,0+1_3\}$
	is defined by the following directed graph: \\

	$$
		\begin{tikzpicture}[baseline=(current axis.outer east)]
			\tikzset{vertex/.style = {shape=circle, draw, minimum size=3.5em}}
			\tikzset{edge/.style = {->,> = latex'}}
			\tikzset{loop/.style = {looseness=6}}

			\node[vertex, minimum size=3.1em] (S_4) at (6, 0) {};

			\node[vertex] (s_0) at (-6, 0) {$0+1_0$};
			\node[vertex] (s_1) at (-3, 0) {$0$};
			\node[vertex] (s_2) at (0, 0) {$0+1_1$};
			\node[vertex] (s_3) at (3, 0) {$0+1_2$};
			\node[vertex] (s_4) at (6, 0) {$0+1_3$};

			\draw[edge] (s_0) to node[midway, above] {$0$} (s_1);
			\draw[edge] (s_1) to node[midway, above] {$0,1$} (s_2);
			\draw[edge] (s_2) to node[midway, above] {$0,1$} (s_3);
			\draw[edge] (s_3) to node[midway, above] {$0,1$} (s_4);

			\draw[edge] (s_0) to[loop above] node[midway] {$0,1$} (s_0);

		\end{tikzpicture}\hfill
	$$

%% Problem #03
\problem{}

	Define a DFA equivalent to the following NFA.

	$$
	\begin{tikzpicture}[baseline=(current axis.outer east)]
		\tikzset{vertex/.style = {shape=circle, draw, minimum size=2.5em}}
		\tikzset{edge/.style = {->,> = latex'}}
		\tikzset{loop/.style = {looseness=6}}

		\node[vertex, minimum size=2.3em] (h) at (4, 0) {};

		\node[vertex] (a) at (-4, 0) {$a$};
		\node[vertex] (b) at (-2, 1) {$b$};
		\node[vertex] (c) at (0, 1) {$c$};
		\node[vertex] (d) at (2, 1) {$d$};
		\node[vertex] (e) at (-2, -1) {$e$};
		\node[vertex] (f) at (0, -1) {$f$};
		\node[vertex] (g) at (2, -1) {$g$};
		\node[vertex] (h) at (4, 0) {$h$};

		\draw[edge] (a) to node[midway, above] {$0$} (b);
		\draw[edge] (b) to node[midway, above] {$0$} (c);
		\draw[edge] (c) to node[midway, above] {$0$} (d);
		\draw[edge] (d) to node[midway, above] {$0$} (h);

		\draw[edge] (a) to node[midway, below] {$1$} (e);
		\draw[edge] (e) to node[midway, below] {$1$} (f);
		\draw[edge] (f) to node[midway, below] {$1$} (g);
		\draw[edge] (g) to node[midway, below] {$1$} (h);

		\draw[edge] (a) to[loop above] node[midway] {$0,1$} (a);
		\draw[edge] (d) to[loop above] node[midway] {$0,1$} (d);
		\draw[edge] (g) to[loop below] node[midway] {$0,1$} (g);
	\end{tikzpicture}
	$$

\solution

	$$
	\begin{tikzpicture}[font=\tiny, baseline=(current axis.outer east)]
		\tikzset{vertex/.style = {shape=circle, draw, minimum size=2.6em}}
		\tikzset{edge/.style = {->,> = latex'}}
		\tikzset{loop/.style = {looseness=6}}

		\node[vertex] (e) at (0,0) {$\epsilon$};

		\node[vertex] (0) at (2,  1.5) {$0$};
		\node[vertex] (1) at (2, -1.5) {$0$};

		\node[vertex] (00) at (4,  3) {$00$};
		\node[vertex] (11) at (4, -3) {$11$};

		\node[vertex] (000) at (6,  4.5) {$000$};
		\node[vertex] (111) at (6, -4.5) {$111$};

		\node[vertex] (0001) at (8,  3) {$0001$};
		\node[vertex] (1110) at (8, -3) {$1110$};

		\node[vertex] (00010) at (10,  4.5) {$00010$};
		\node[vertex] (00011) at (10,  1.5) {$00011$};
		\node[vertex] (11100) at (10, -1.5) {$11100$};
		\node[vertex] (11101) at (10, -4.5) {$11101$};

		\node[vertex] (111000) at (12,  0) {$111000$};
		\node[vertex] (F) at (14, 0) {$F$};

		%% Final states
		\node[vertex, minimum size=2.2em] (fs_11101) at (10, -4.5) {};
		\node[vertex, minimum size=2.2em] (fs_00010) at (10,  4.5) {};
		\node[vertex, minimum size=2.2em] (fs_F) at (14, 0) {};

		%% Self loops for final states
		\draw[edge] (00010) to[loop above] node[midway] {$0$} (00010);
		\draw[edge] (11101) to[loop below] node[midway] {$1$} (11101);
		\draw[edge] (F) to[loop right] node[midway] {$0,1$} (F);

	\end{tikzpicture}
	$$

\end{document}
