\documentclass[12pt]{jhwhw}
\author{Ian Malerich}
\title{Com S 331: Homework 6}
\usepackage{amssymb, amsfonts, amsthm, mathtools, graphicx, breqn}
\usepackage{subfig, float, scrextend, setspace, soul}
\usepackage{tikz, subcaption, caption}
\usetikzlibrary{arrows.meta, automata, positioning}

\onehalfspacing
\begin{document}
\raggedright

%% Problem 1
\problem{}

	Define a context-free grammar for the language L = $\{0^n1^m0^m1^n: n,m\in \mathbb{N}\}$.

\solution

	Assumes $0\in \mathbb{N}$. \\
	G = $(\{S, X\}, \{0,1\}, P, S)$ with productions P given by

	\begin{align*}
		S &\rightarrow 0S1 \\
		S &\rightarrow \epsilon  \\
		S &\rightarrow X \\
		X &\rightarrow 1X0 \\
		X &\rightarrow \epsilon \\
	\end{align*}

	Here, $S$ is used to create the string $0^n1^n$. Once these have been created,
	$S$ transfers control to $X$ who creates $1^m0^m$ exactly in the middle of the
	string created by $S$. When put together we end up with $0^n1^m0^m1^n$.
	Note that $0^n1^n$ can be created by sending $S\rightarrow\epsilon$ and the string
	$1^m0^m$ can be created by immediately sending $S\rightarrow X$.

%% Problem 2
\problem{}

	Define a context-free grammar for the language L = $\{a^nb^m: n\leq 3m\}$.

\solution

	Assumes $n,m\in \mathbb{N} \text{ and } 0\in \mathbb{N}$.  \\
	G = $(\{S\}, \{a,b\}, P, S)$ with productions $P$ given by

	\begin{align*}
		S &\rightarrow Sb \\
		S &\rightarrow aSb \\
		S &\rightarrow aaSb\\
		S &\rightarrow aaaSb\\
		S &\rightarrow \epsilon \\
	\end{align*}

	My approach here is pretty simple, first note that we can have any number of $a$'s that 
	we want so long as we don't have more than 3 for every $b$ at the end of the string.
	Thus we can say that for every b, there are no more than 3$a$'s present in the string.
	This is exactly what my grammar shows, along with the requirement that all $a$'s precede
	all $b$'s. Note that the empty string is of course allowed,
	with $n=m=0$.

%% Problem 3
\problem{}

	Define a CFG that generates the following language over $\{t,f,\wedge,\vee,\neg,(,),=\}$: \\
	L = $\{w=x : w \text{ is a logical expression over } \{t,f\}, x\in \{t,f\}$
	$\text{ and } x \text{ is the truth value of } w\}$

\solution

	G = $\{\{S,T,F\}, \{t,f,\wedge,\vee,\neg,(,),=\}, P, S\}$ with productions $P$ given by

	\begin{align*}
		S &\rightarrow T=t \\
		S &\rightarrow F=f \\
		T &\rightarrow t \\
		F &\rightarrow f \\
		T &\rightarrow \neg(F) \\
		F &\rightarrow \neg(T) \\
		T &\rightarrow (T \wedge T) \\
		F &\rightarrow (T \wedge F) \\
		F &\rightarrow (F \wedge T) \\
		F &\rightarrow (F \wedge F) \\
		T &\rightarrow (T \vee T) \\
		T &\rightarrow (T \vee F) \\
		T &\rightarrow (F \vee T) \\
		F &\rightarrow (F \vee F) \\
	\end{align*}

	The general idea with this CGF is that we start with the solution (true or false) and
	then derive what strings can produce that result from the definitions of each operation.
	For example $T\rightarrow \neg(F)$ and $T\rightarrow \neg(T)$ describe what happens
	via the negation operator on each possible input, where the variables $T$ and $F$ stand
	for a sort of ``partial sum" for true and false respectively.

\end{document}
