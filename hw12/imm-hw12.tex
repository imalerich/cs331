\documentclass[12pt]{jhwhw}
\author{Ian Malerich}
\title{Com S 331: Homework 11}
\usepackage{amssymb, amsfonts, amsthm, mathtools, graphicx, breqn}
\usepackage{subfig, float, scrextend, setspace, soul}
\usepackage{tikz, subcaption, caption, graphviz, color}
\usetikzlibrary{arrows.meta, automata, positioning}

\newcommand{\hlred}[1]{{\sethlcolor{pink}\hl{#1}}}
\newcommand{\hlgreen}[1]{{\sethlcolor{lime}\hl{#1}}}

\onehalfspacing
\begin{document}
\raggedright

%% Problem 1
\problem{}

	You are employed as a programmer, and you are asked to write the given program.
	For each case, write "Y" or "N" if the program can or cannot 
	be written. Do not concern yourself with memory limitations, that is, assume
	that the computer used to run your program has a large, effectively unbounded,
	amount of memory.

\solution

\part %% (a)
	\begin{center}
	\textit{
		receives in input a generic C program $x$, 
		and counts the number of statements in $x$.
	}
	\end{center}
	Y

\part %% (b)
	\begin{center}
	\textit{
		receives in input a generic C program $x$,
		and an input string $w$, and counts the number statements executed
		at least once when $x$ runs on input $w$.
	}
	\end{center}
	N

\part %% (c)
	\begin{center}
	\textit{
		receives in input a generic C program $x$ and an input string  $w$, and 
		counts the number of statements never executed when $x$ runs on
		input $w$.
	}
	\end{center}
	N

\part %% (d)
	\begin{center}
	\textit{
		receives an input a generic C program $x$ and decides whether $x$ is
		syntactically correct.
	}
	\end{center}
	Y

\part %% (e)
	\begin{center}
	\textit{
		receives in input two natural numbers and computes a specific function
		$f: \mathbb{N}^2 \Rightarrow \mathbb{N}$.
	}
	\end{center}
	N

\part %% (f)
	\begin{center}
	\textit{
		receives in input a generic arithmetic expression $e$ composed of integers
		and the four arithmetic operators, and computes its value.
	}
	\end{center}
	Y

\part %% (g)
	\begin{center}
	\textit{
		halts on the empty string
	}
	\end{center}
	Y

\part %% (h)
	\begin{center}
	\textit{
		receives in input a generic C program $x$ and decides whether $x$ halts
		only on the empty string.
	}
	\end{center}
	N

\part %% (i)
	\begin{center}
	\textit{
		receives in input two generic regular expressions and decides whether
		they are equivalent.
	}
	\end{center}
	Y

\part %% (j)
	\begin{center}
	\textit{
		receives in input a generic C program $x$ and the name of one of its functions, f,
		and decides whether $x$ can ever call $f$.
	}
	\end{center}
	N

\part %% (k)
	\begin{center}
	\textit{
		receives in input a generic C program $x$, an input string $w$, and the name
		of one of its functions, $f$, and decides whether $x$ calls $f$ when
		running on input $w$.
	}
	\end{center}
	N

\part %% (l)
	\begin{center}
	\textit{
		receives in input two generic C program $x_1$ and $x_2$ and an input
		string $w$, and decides whether $x_1$ and $x_2$ produce the same output
		when running on input $w$.
	}
	\end{center}
	N

\part %% (m)
	\begin{center}
	\textit{
		receives in input two generic C programs $x_1$ and $x_2$ and decides
		whether $x_1$ and $x_2$ produce the same output when running on every
		possible input.
	}
	\end{center}
	N

\part %% (n)
	\begin{center}
	\textit{
		receives in input two generic C programs $x_1$ and $x_2$, and decides
		whether $x_1$ and $x_2$ produce the same output when running on at least one
		input.
	}
	\end{center}
	N

\part %% (o)
	\begin{center}
	\textit{
		receives in input a generic C program $x$, an input string $w$, and a natural
		number $n$, and decides whether $x$ uses less than $n$ bytes of memory when 
		running on $w$.
	}
	\end{center}
	Y

\part %% (p)
	\begin{center}
	\textit{
		receives in input a generic C program $x$, an input string $w$, and 
		decides whether there is an $n\in \mathbb{N}$ such that $x$ uses less
		than $n$ bytes of memory running on input $w$.
	}
	\end{center}
	N

%% Problem 2
\problem{}

	Use reduction to prove that the language
	$$
		L = \{p(M)p(w) :\text{ the TM $M$ never enters its initial state again when
			running on $w$}\}
	$$
	is undecidable.

\solution

	TODO

\end{document}
